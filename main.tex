\documentclass[amsmath,amssymb,notitlepage,11pt,reprint,aip]{revtex4-1}

%\bibliographystyle{apsrev4-1}
\usepackage{graphicx}% Include figure files
\usepackage{dcolumn}% Align table columns on decimal point
\usepackage{bm}% bold math
\usepackage[colorlinks = true,
            linkcolor = red,
            urlcolor  = black,
            citecolor = red,
            anchorcolor = black]{hyperref}
%\usepackage[mathlines]{lineno}% Enable numbering of text and display math
%\linenumbers\relax % Commence numbering lines
\usepackage{lipsum}
\usepackage[table,xcdraw]{xcolor}
\usepackage{multirow}
\usepackage{soul}
\usepackage{simpler-wick}
\usepackage{chemformula}
\usepackage{xr}
\usepackage{hhline}
\usepackage{braket}
\usepackage[capitalize]{cleveref}% add hypertext capabilities
\usepackage{listings}
\usepackage{xcolor}
\usepackage{xspace}

\newcommand{\ahat}{\hat{a}}
\newcommand{\bvec}[1]{\bm{\mathrm{#1}}}
\newcommand{\adj}{^{\dagger}}
\newcommand{\pz}{\Phi_0}
\newcommand{\barh}{\bar{H}}
\newcommand{\dl}{\mathrm{d}}


\definecolor{goodorange}{RGB}{225,125,0}
\definecolor{goodgreen}{RGB}{0,125,0}
\definecolor{goodred}{RGB}{220,50,25}
\definecolor{goodblue}{RGB}{25,25,150}

\newcommand{\note}[2]{
\ifthenelse{\equal{#1}{F}}{
\colorbox{goodorange}{\textcolor{white}{\footnotesize \fontfamily{phv}\selectfont #1}}
    \textcolor{goodorange}{{\footnotesize \fontfamily{phv}\selectfont #2}}\xspace
}{}
\ifthenelse{\equal{#1}{S}}{
\colorbox{goodred}{\textcolor{white}{\footnotesize \fontfamily{phv}\selectfont #1}}
    \textcolor{goodred}{{\footnotesize \fontfamily{phv}\selectfont #2}}\xspace
}{}
\ifthenelse{\equal{#1}{Z}}{
\colorbox{goodgreen}{\textcolor{white}{\footnotesize \fontfamily{phv}\selectfont #1}}
    \textcolor{goodgreen}{{\footnotesize \fontfamily{phv}\selectfont #2}}\xspace
}{}
}
\definecolor{codegreen}{rgb}{0,0.6,0}
\definecolor{codegray}{rgb}{0.5,0.5,0.5}
\definecolor{codepurple}{rgb}{0.58,0,0.82}
\definecolor{backcolour}{rgb}{0.95,0.95,0.92}

\lstdefinestyle{mystyle}{
    backgroundcolor=\color{backcolour},   
    commentstyle=\color{codegreen},
    keywordstyle=\color{magenta},
    numberstyle=\tiny\color{codegray},
    stringstyle=\color{codepurple},
    basicstyle=\ttfamily\footnotesize,
    breakatwhitespace=false,         
    breaklines=true,                 
    captionpos=b,                    
    keepspaces=true,                 
    numbers=left,                    
    numbersep=5pt,                  
    showspaces=false,                
    showstringspaces=false,
    showtabs=false,                  
    tabsize=2
}

\lstset{style=mystyle}


%\usepackage[showframe,%Uncomment any one of the following lines to test 
%%scale=0.7, marginratio={1:1, 2:3}, ignoreall,% default settings
%%text={7in,10in},centering,
%%margin=1.5in,
%%total={6.5in,8.75in}, top=1.2in, left=0.9in, includefoot,
%%height=10in,a5paper,hmargin={3cm,0.8in},
%]{geometry}

\begin{document}

\preprint{APS/123-QED}

\title{Notes on spectroscopy}

\author{Zijun Zhao}
\email{brian.zhaozijun@gmail.com}
\affiliation{Department of Chemistry and Cherry Emerson Center for Scientific Computation, Emory University, Atlanta, Georgia, 30322, United States}

\date{\today}


\maketitle
\section{Multipole expansions}
This section is adapted from Chap.~2 of Barron's \textit{Molecular Light Scattering and Optical Activity}.\cite{barron.2004.10.1017/CBO9780511535468}

A collection of point charges has multipole moments.
The $n$-th order multipole moment is defined as
\begin{equation}
    \xi_{\alpha\beta\dots\nu}^{(n)}=\frac{(-1)^n}{n!}\sum_{i}e_ir_i^{2n+1}\nabla_{i\alpha}\nabla_{i\beta}\dots\nabla_{i\nu}\left(\frac{1}{r_i}\right),
\end{equation}
The zeroth moment is the net charge:
\begin{equation}
    q = \sum_ie_i.
\end{equation}
The first moment is the dipole moment vector:
\begin{equation}
    \bvec{\mu} = \sum_ie_i\bvec{r}_i.
\end{equation}
The dipole moment is origin dependent if the system is not charge neutral: moving the origin $\bvec{O}$ to $\bvec{O}+\bvec{a}$ changes the dipole moment to
\begin{equation}
    \bvec{\mu}' = \sum_ie_i(\bvec{r}_i-\bvec{a}) = \bvec{\mu} - q\bvec{a}.
\end{equation}
The second moment is the quadrupole moment tensor:
\begin{equation}
    \Theta=\frac{1}{2}\sum_ie_i\left(3\bvec{r}_i\bvec{r}_i-r_i^2\bvec{I}\right),
\end{equation}
where $\bvec{r}_i\bvec{r}_i$ is the outer product, and $r_i^2$ is the dot product $\bvec{r}_i\cdot\bvec{r}_i$.
In Cartesian tensor notation, the quadrupole moment tensor is
\begin{equation}
    \Theta_{\alpha\beta}=\frac{1}{2}\sum_ie_i\left(3r_{i\alpha}r_{i\beta}-r_i^2\delta_{\alpha\beta}\right).
\end{equation}
The $\Theta_{\alpha\beta}$ tensor is a symmetric rank-2 tensor, with zero trace, hence has 5 independent components.
Likewise, the quadrupole moment tensor is origin dependent if the net charge and dipole moment are nonzero.

An electric field is generated by a charge distribution $\rho(\bvec{r})$:
\begin{equation}
    \rho(\bvec{r}) = \sum_ie_i\delta(\bvec{r}-\bvec{r}_i).
\end{equation}
The corresponding potential can be found by solving the Poisson equation for static sources:
\begin{equation}
    \nabla^2\phi=-\frac{\rho}{\epsilon\epsilon_0},
\end{equation}
whose solution is
\begin{equation}
    \phi(\bvec{R}) = \frac{1}{4\pi\epsilon\epsilon_0}\int\frac{\rho(\bvec{r})\dl V}{|\bvec{R}-\bvec{r}|}.
\end{equation}
So in this case, the potential becomes
\begin{equation}
    \phi(\bvec{R}) = \frac{1}{4\pi\epsilon\epsilon_0}\sum_i\frac{e_i}{|\bvec{R}-\bvec{r}_i|}.
\end{equation}
Assuming $|\bvec{R}|\gg|\bvec{r}_i|$, we can write down the following Taylor series:
\begin{align}
    \frac{1}{|\bvec{R}-\bvec{r}_i|} &= (R_{\alpha}R_{\alpha}-2R_{\alpha}r_{i\alpha}+r_{i\alpha}r_{i\alpha})^{-1/2} \\
    &= \frac{1}{R} + \frac{R_{\alpha}r_{i\alpha}}{R^3} + \frac{1}{2}\left(\frac{3R_{\alpha}r_{i\alpha}r_{i\beta}R_{\beta}}{R^5}-\frac{r_{i}^2}{R^3}\right) + \dots.\nonumber
\end{align}

\bibliography{main}
\end{document}
